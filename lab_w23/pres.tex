%%%%%%%%%%%%%%%%%%%%
% Header Stuff
%%%%%%%%%%%%%%%%%%%%
\documentclass[12pt]{article}
\usepackage[utf8]{inputenc}
\usepackage[margin=0.5in]{geometry}
\usepackage{setspace,graphicx,multirow,cite}

% Title/Sections Package
\usepackage{titlesec}
\titleformat*{\section}{\normalsize\bfseries}
\titleformat*{\subsection}{\normalsize\bfseries}
\titleformat*{\subsubsection}{\normalsize\bfseries}
\titleformat*{\paragraph}{\normalsize\itshape}
\titleformat*{\subparagraph}{\normalsize\itshape}
\titlespacing*{\section}{0pt}{1em}{0pt}
\titlespacing*{\subsection}{0pt}{1em}{0pt}
\titlespacing*{\subsubsection}{0pt}{1em}{0pt}


\def\blank{\medskip\hrule\medskip}
\setlength\parindent{0pt}

\usepackage{amsthm}
\newtheorem{definition}{Definition}

\usepackage{xcolor,tikz}
\usepackage{enumitem,float}
\usepackage{wasysym}
\usepackage[linguistics]{forest}
\usepackage{gb4e}


% Title
\title{{\normalsize\bfseries {Prediction of Japanese Q-Particles}}}
\author{\normalsize\bfseries {}}
\date{\vspace{-10ex}}


%%%%%%%%%%%%%%%%%%%%
% Main Document
%%%%%%%%%%%%%%%%%%%%
\begin{document}

\usetikzlibrary{topaths}
\tikzstyle{every picture}+=[remember picture,inner xsep=0,inner ysep=0.25ex]


\maketitle

\section{A Motivating Question}
\textbf{Consider the following sentence:}
\begin{center}
    ``He has (really) (never) (...) \textcolor{blue}{gone}/\textcolor{red}{*go}/\textcolor{red}{*went} to Europe.''
\end{center}

\begin{center}
    It may be unsurprising to see ``gone'' after the occurrence of ``has''.\\
    But, when you see ``has'', do you predict something upcoming?\\
    The claim of some (and myself): \textbf{Yes}.
\end{center}


\textbf{If so:}
\begin{itemize}
    \item What exactly do you predict? (``gone''? past-participle in general?)
    \item When do you predict it? (when you reach ``has'', or?)
    \item How do you predict it? (what knowledge informs your prediction?)
\end{itemize}

\vspace{0.2cm}

\textbf{A potential answer to these questions can be given at the level of syntax:}

\begin{itemize}%[leftmargin=1em]
    \renewcommand{\labelitemi}{$\Rightarrow$}
    \item We predict structures by using structural information and deductions therefrom. (how/what)
    \item This information is encoded at some trigger point in the input. (when)
\end{itemize} 

\vspace{0.2cm}
\begin{center}
    \textbf{But is this the case? Let's test...}
\end{center}


\blank

\section{Some Japanese Data}
In Japanese, wh-phrases must be c-commanded by 
-ka or -no (a question particle), in the C position.

\begin{exe}
    \ex{
    \begin{xlist}
        \ex{ No c-commanding QP  $\rightarrow$ BAD
            \gll *\textcolor{red}{Dare-ga} [Hanako-ga ringo-o tabeta-ka] tazuneta. \\
            Who-nom Hanako-nom apple-acc ate-Q asked \\
            Intended: Who asked if Hanako ate an apple?}\\
        \ex{ (Matrix Wh) Wh is c-commanded by QP $\rightarrow$ GOOD
        \gll \textcolor{red}{Dare-ga} [Hanako-ga ringo-o tabeta-ka] tazuneta-\textcolor{red}{no}. \\
             Who-nom Hanako-nom apple-acc ate-Q asked-Q \\
             Who asked if Hanako ate an apple?}\\
        \ex{ (Embedded Wh) Wh is c-commanded by QP $\rightarrow$ GOOD
        \gll Hanako-ga [Keiko-ga \textcolor{red}{nani-o} tabeta-to] itta-\textcolor{red}{no}. \\
                Hanako-nom Keiko-nom what-acc ate-Q said-Q \\
                What did Hanako say that Keiko ate?}
    \end{xlist}}
\end{exe}


**Two wh-questions (matrix and embedded) can be licensed by a single matrix QP!**
\begin{exe}    
    \ex{
        \gll \textcolor{red}{Dono} gakusei-ga [kyoushi-ga \textcolor{red}{nani-o} chuumon-shita-to]
              kuwashi-ku tazunemashita-\textcolor{red}{ka}\\
        Which student-nom teacher-nom what-acc order-did-dec in-detail asked-Q \\
        Which student asked what the teacher ordered in detail?}
        \label{ex:twowh}
\end{exe}


\begin{center}
    Example \ref{ex:twowh} presents an ambiguity to the parser.
\end{center}
\vspace{0.2cm}
\fbox{%
    \parbox{\textwidth}{%
    \begin{center}
        ``\textcolor{red}{Dono} gakusei-ga kyoushi-ga \textcolor{red}{nani-o}...'' $\rightarrow$ is it 1 or 2 QPs coming?\\
        \vspace{0.3cm}
        chuumon-shita-to kuwashi-ku tazunemashita-\textcolor{red}{ka}'' \\(Just matrix QP)\\ 
        \vspace{0.2cm}
        chuumon-shita-\textcolor{red}{ka} kuwashi-ku tazunemashita-\textcolor{red}{ka}''\\ (Both matrix and embedded QP)
    \end{center}
    }%
}

\vspace{0.2cm}
\begin{center}
    \textbf{We can use this ambiguity to test for prediction.}
\end{center}

\blank

\section{An Experiment}
Using the data from above we can make an experiment to test for predictive effects.
Assume the following conditions and an example item. Wh quetions and QPs are highlighted red.

\begin{table}[H]
    \centering
    \begin{tabular}{cl|ll|}
    \cline{3-4}
    \multicolumn{1}{l}{}                               &      & \multicolumn{2}{c|}{\textbf{F1}} \\ \cline{3-4} 
    \textbf{}                                          &      & \multicolumn{1}{l|}{1QP}   & 2QP  \\ \hline
    \multicolumn{1}{|c|}{\multirow{2}{*}{\textbf{F2}}} & Dono (Which)   & \multicolumn{1}{l|}{Cond. A }    & Cond. B    \\ \cline{2-4} 
    \multicolumn{1}{|c|}{}                             & Sono (That) & \multicolumn{1}{l|}{Cond. C}    & Cond. D    \\ \hline
    \end{tabular}
    \caption{Conditions for Experiment}
    \label{table:exp2}
\end{table}


\begin{exe}    
    \ex{
        \begin{xlist}
        \ex{
        \gll \textcolor{red}{Dono} gakusei-ga [kyoushi-ga \textcolor{red}{nani-o} chuumon-shita-to]
              kuwashi-ku tazunemashita-\textcolor{red}{ka}\\
        Which student-nom teacher-nom what-acc order-did-dec in-detail asked-Q \\
        Which student asked what the teacher ordered in detail?}

        \vspace{0.2cm}

        \ex{
        \gll \textcolor{red}{Dono} gakusei-ga [kyoushi-ga \textcolor{red}{nani-o} chuumon-shita-\textcolor{red}{ka}]
              kuwashi-ku tazunemashita-\textcolor{red}{ka}\\
        Which student-nom teacher-nom what-acc order-did-Q in-detail asked-Q \\
        Which student asked what the teacher ordered in detail?}

        \vspace{0.2cm}

        \ex{
        \gll Sono gakusei-ga [kyoushi-ga \textcolor{red}{nani-o} chuumon-shita-to]
              kuwashi-ku tazunemashita-\textcolor{red}{ka}\\
        That student-nom teacher-nom what-acc order-did-dec in-detail asked-Q \\
        What did that student ask in detail that the teacher ate?}
        
        \vspace{0.2cm}

        \ex{
        \gll Sono gakusei-ga [kyoushi-ga \textcolor{red}{nani-o} chuumon-shita-\textcolor{red}{ka}]
              kuwashi-ku tazunemashita-\textcolor{red}{ka}\\
        That student-nom teacher-nom what-acc order-did-Q in-detail asked-Q \\
        What did that student ask in detail that the teacher ate?}
        \end{xlist}
    }
\end{exe}

\vspace{0.2cm}

\textbf{Hypothesis: The human sentence processor is structurally predictive.}
\begin{itemize}
    \item When? $\rightarrow$ Upon reaching each Wh (dono, nani-o)
    \item What? $\rightarrow$ An upcoming QP in the first possible C position
    \item How? $\rightarrow$ Employing (structural) knowledge that Wh questions are licensed by c-commanding QP in C position
\end{itemize}

\vspace{0.2cm}

\section{Expected Results}
\textbf{Under the predictive hypothesis, we expect some things to happen for conditions A and B.}
\begin{enumerate}
    \item Matrix ``Dono'' triggers prediction of upcoming QP in Matrix C position.
    \begin{center}
        [_{CP} \tikz[baseline=(node1.base)]\node (node1) {Dono};
        ...\tikz[baseline=(node2.base)]\node (node2) {-ka};]
    \end{center}

    \item Embedded ``nani-o'' triggers prediction of upcoming QP in (some) C position.\\
    \begin{center}
        [_{CP} \tikz[baseline=(node3.base)]\node (node3) {Dono}; 
               gakusei-ga[_{CP} kyoushi-ga nani-o...]...
               \tikz[baseline=(node4.base)]\node (node4) {-ka};]
    \end{center}

    \item Given the QP predicted by ``Dono'', the least effort is to assume nani-o ``covered'' 
          by matrix QP. This is the case of condition A.\\
    \begin{center}
        [_{CP} \tikz[baseline=(node5.base)]\node (node5) {Dono};
                gakusei-ga[_{CP} kyoushi-ga \tikz[baseline=(node6.base)]\node (node6) {nani-o};...]
                ...\tikz[baseline=(node7.base)]\node (node7) {-ka};]
    \end{center}

    \item If an embedded QP appears, we should observe slowdown at that point.
          This is the case of condition B.\\
    \begin{center}
        [_{CP} \tikz[baseline=(node8.base)]\node (node8) {Dono};
               gakusei-ga[_{CP} kyoushi-ga \tikz[baseline=(node9.base)]\node (node9) {nani-o}; 
               chuumon-shita\tikz[baseline=(node10.base)]\node (node10) {-ka};]
               ...\tikz[baseline=(node11.base)]\node (node11) {-ka};]
    \end{center}
\end{enumerate}

\begin{tikzpicture}[overlay]
    \draw[-latex,red] (node2.north) to[bend right] (node1.north); % 1
    \draw[-latex,red] (node4.north) to[bend right=15] (node3.north); %2
    \draw[-latex,red] (node7.north) to[bend right=15] (node5.north); %3
    \draw[-latex,red] (node7.north) to[bend right=15] (node6.north); %3
    \draw[-latex,red] (node11.north) to[bend right=15] (node8.north); %4
    \draw[-latex,red] (node10.north) to[bend right=15] (node9.north); %4
\end{tikzpicture}

\begin{center}
\textbf{In condition B, a second (embedded) QP shows up. \\ We should observe a significant slowdown at that point compared to A.}
\end{center}


\blank

\section{Why Important?}
Showing significant slowdown at the embedded QP in Condition B is in accordance with our predictive hypothesis.\\

What would other hypotheses say about the reading time at the embedded QP?

\begin{itemize}
    \item Integration: No slowdown is anticipated since the embedded QP should fit very well into the context of
    a nearby Wh which it can license.
    \item Others?
\end{itemize}

This experiment can successfully separate the predictions of these differing hypotheses. This experimental
framework can be used in other studies to disambiguate competing theories of prediction, integration, etc.
This has implications for theoretical space of possible parsers (i.e., what capabilities the parser should have).

\end{document}
