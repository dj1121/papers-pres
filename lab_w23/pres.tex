%%%%%%%%%%%%%%%%%%%%
% Header Stuff
%%%%%%%%%%%%%%%%%%%%
\documentclass[12pt]{article}
\usepackage[utf8]{inputenc}
\usepackage[margin=0.3in]{geometry}
\usepackage{setspace,graphicx,multirow,cite}

% Title/Sections Package
\usepackage{titlesec}
\titleformat*{\section}{\normalsize\bfseries}
\titleformat*{\subsection}{\normalsize\bfseries}
\titleformat*{\subsubsection}{\normalsize\bfseries}
\titleformat*{\paragraph}{\normalsize\itshape}
\titleformat*{\subparagraph}{\normalsize\itshape}
\titlespacing*{\section}{0pt}{1em}{0pt}
\titlespacing*{\subsection}{0pt}{1em}{0pt}
\titlespacing*{\subsubsection}{0pt}{1em}{0pt}

% Linguistics Packages
\usepackage{xcolor}
\usepackage{enumitem,float}
\usepackage[linguistics]{forest}
\usepackage{gb4e}


% Title
\title{{\normalsize\bfseries {Prediction of Japanese Q-Particles}}}
\author{\normalsize\bfseries {}}
\date{}


%%%%%%%%%%%%%%%%%%%%
% Main Document
%%%%%%%%%%%%%%%%%%%%
\begin{document}

\maketitle

\section{Detecting Structural Prediction}
My work has been concerned with structural prediction by the parser. Namely,
\textbf{what}, \textbf{when}, \textbf{why}? But verifying that the parser is performing 
structural prediction in online processing is *hard*. Several reasons:

\begin{itemize}
    \item Surprise ($>$RT) and facilitation ($<$RT) effects often have multiple explanations.
    \item A common explanation is that predictive effects are actually due to integration.
\end{itemize}

\noindent
Consider the following example:
\begin{exe}
    \ex
    \begin{xlist}
        \ex $AuxP \rightarrow [Aux \hspace{0.1cm} VP]$
        \ex $VP \rightarrow V'$
        \ex $ V' \rightarrow  [(Adv) \hspace{0.1cm} V]$
        \ex $V'\rightarrow [V \hspace{0.1cm} NP]$
    \end{xlist}
\end{exe}
``He has (really) (never) (...) gone to Europe.''\\

\noindent
Consider we observe a facilitation at \textit{gone}. Is it due to parser's prediction triggered by
\textit{has}? Or, is it because \textit{gone}....? Need an experiment with prediction and integration
saying different things.



\section{Wh-Licensing w/ Q-particles}
In Japanese wh-interrogative sentences, a wh-phrase must be licensed (c-commanded)
by a verbal suffix -ka or -no (a Question-particle), which is located in the complementizer
position:

\begin{exe}
    \ex{
    \begin{xlist}
        \ex{ No c-commanding QP  $\rightarrow$ BAD
            \gll *\textcolor{red}{Dare-ga} [Hanako-ga ringo-o tabeta-ka] tazuneta. \\
            Who-nom Hanako-nom apple-acc ate-Q asked \\
            Intended: Who asked if Hanako ate an apple?}\\
        \ex{ (Matrix) Wh is c-commanded by QP $\rightarrow$ GOOD
        \gll \textcolor{red}{Dare-ga} [Hanako-ga ringo-o tabeta-ka] tazuneta-\textcolor{red}{no}. \\
             Who-nom Hanako-nom apple-acc ate-Q asked-Q \\
             Who asked if Hanako ate an apple?}\\
        \ex{ (Embedded) Wh is c-commanded by QP $\rightarrow$ GOOD
        \gll Hanako-ga [Keiko-ga \textcolor{red}{nani-o} tabeta-to] itta-\textcolor{red}{no}. \\
                Hanako-nom Keiko-nom what-acc ate-Q said-Q \\
                What did Hanako say that Keiko ate?}
    \end{xlist}}
\end{exe}

\noindent
**In addition, two wh-questions (matrix and embedded) can be licensed by a single QP!**
\begin{exe}    
    \ex{
        \gll \textcolor{red}{Dono} gakusei-ga [kyoushi-ga \textcolor{red}{nani-o} chuumon-shita-to]
              kuwashi-ku tazunemashita-\textcolor{red}{ka}\\
        Which student-nom teacher-nom what-acc order-did-dec in-detail asked-Q \\
        Which student asked what the teacher ordered in detail?}\\
        \label{ex:twowh}
\end{exe}

\section{Potential for Prediction Effects}
Example \ref{ex:twowh} presents an \textbf{ambiguity at the embedded Wh} to the parser.
\begin{center}
``\textcolor{red}{Dono} gakusei-ga kyoushi-ga \textcolor{red}{nani-o}...'' $\rightarrow$ 1 or 2 QPs coming?
\end{center}
Either we have 1 QP which licenses both \textit{Dono} and \textit{nani-o} 
or we have 2QPs, each within the CP of each Wh question. As below: \\

\noindent\fbox{%
    \parbox{\textwidth}{%
    \begin{center}
        ``chuumon-shita-to kuwashi-ku tazunemashita-\textcolor{red}{ka}'' (Just matrix QP)\\ 
        --OR-- \\
        ``chuumon-shita-\textcolor{red}{ka} kuwashi-ku tazunemashita-\textcolor{red}{ka}'' (Both matrix and embedded QP)
    \end{center}
    }%
}

\begin{itemize}
    \item It is presumed that the parser picks the option of 1QP as this prevents it from doing extra work to resolve
    the licensing constraint.
    \item We should, as a result of this prediction, observe surprise if we come across an embedded QP.
\end{itemize}

\section{Prediction vs. Integration}
Let us compare what a predictive theory and an integrative theory would say about such items.








\end{document}
