%%%%%%%%%%%%%%%%%%%%
% Header Stuff
%%%%%%%%%%%%%%%%%%%%
\documentclass[12pt]{article}
\usepackage[utf8]{inputenc}
\usepackage[margin=0.5in]{geometry}
\usepackage{setspace,graphicx,multirow,cite}

% Title/Sections Package
\usepackage{titlesec}
\titleformat*{\section}{\normalsize\bfseries}
\titleformat*{\subsection}{\normalsize\bfseries}
\titleformat*{\subsubsection}{\normalsize\bfseries}
\titleformat*{\paragraph}{\normalsize\itshape}
\titleformat*{\subparagraph}{\normalsize\itshape}
\titlespacing*{\section}{0pt}{1em}{0pt}
\titlespacing*{\subsection}{0pt}{1em}{0pt}
\titlespacing*{\subsubsection}{0pt}{1em}{0pt}


\def\blank{\medskip\hrule\medskip}
\setlength\parindent{0pt}

\usepackage{amsthm}
\newtheorem{definition}{Definition}

\usepackage{xcolor,soul}
\usepackage{tikz}
\usepackage{enumitem,float}
\usepackage{wasysym}
\usepackage[linguistics]{forest}
\usepackage[normalem]{ulem}
\usepackage{gb4e}

\newcommand*\circled[1]{\tikz[baseline=(char.base)]{
            \node[shape=circle,draw,inner sep=1pt] (char) {#1};}}

% Title
\title{{\normalsize\bfseries {Japanese Q-Particles \& Predictive Parsing}}}
\author{\normalsize\bfseries {}}
\date{\vspace{-10ex}}


%%%%%%%%%%%%%%%%%%%%
% Main Document
%%%%%%%%%%%%%%%%%%%%
\begin{document}

\maketitle

\section{A Motivating Question}
\textbf{Consider the following sentence:}
\begin{center}
    ``He has (really) (never) (...) \textcolor{blue}{gone}/\textcolor{red}{*go}/\textcolor{red}{*went} to Europe.''
\end{center}

\begin{center}
    Let's say you observe reading speedup at \textit{gone}.\\
    Why?\\
\end{center}

\textbf{One possibility = You predicted it}
\begin{itemize}
    \item What = VP_{+perf}
    \item When = Reaching ``has''
    \item How = TBD
\end{itemize}

\begin{figure}[h!]
    \centering
    \begin{forest}
        [TP
            [DP
                [He,roof]
            ]
            [T'
                [T [\textcolor{red}{has},name=src]]
                [VP_{+perf},draw,red]
            ]
            ]
        ]
    \end{forest}
\end{figure}

\textbf{Implications:}
You have to know \textbf{a lot} to make good predictions
\begin{itemize}
    \item Parser is defined on categorical constituents (VP, DP, etc.) (a \textit{knowledge} claim)
    \item Parser has access to relations between those constituents (X controls Y) (a \textit{knowledge} claim)
    \item When seeing one part of a relation, it has a subprocess to predict the unseen part (a \textit{behaviour} 
    claim)
    \item And more...
\end{itemize}

\vspace{0.2cm}
\begin{center}
    \textbf{Is the above (or something like it) the case?}
\end{center}


\blank

\section{Some Japanese Data}
In Japanese, whs must be c-commanded by 
-ka or -no (a question particle) in the C position.

\begin{exe}
    \ex{
    \begin{xlist}

        \sethlcolor{red}
        
        \ex{ No c-commanding QP  $\rightarrow$ \hl{BAD}
            \gll *\textcolor{red}{Dare-ga} [Hanako-ga ringo-o tabeta-ka] tazuneta. \\
            Who-nom Hanako-nom apple-acc ate-Q asked \\
            Intended: Who asked if Hanako ate an apple?}\\

        \sethlcolor{green}

        \ex{ (Matrix Wh) Wh is c-commanded by QP $\rightarrow$ \hl{GOOD}
        \gll \textcolor{red}{Dare-ga} [Hanako-ga ringo-o tabeta-ka] tazuneta-\textcolor{red}{no}. \\
             Who-nom Hanako-nom apple-acc ate-Q asked-Q \\
             Who asked if Hanako ate an apple?}\\
        \ex{ (Embedded Wh) Wh is c-commanded by QP $\rightarrow$ \hl{GOOD}
        \gll Hanako-ga [Keiko-ga \textcolor{red}{nani-o} tabeta-to] itta-\textcolor{red}{no}. \\
                Hanako-nom Keiko-nom what-acc ate-Q said-Q \\
                What did Hanako say that Keiko ate?}
    \end{xlist}}
\end{exe}


**Two wh-questions (matrix and embedded) can be licensed by a single matrix QP!**
\begin{exe}    
    \ex{
        \gll \textcolor{red}{Dono} gakusei-ga [kyoushi-ga \textcolor{red}{nani-o} chuumon-shita-to]
              kuwashi-ku tazunemashita-\textcolor{red}{ka}\\
        Which student-nom teacher-nom what-acc order-did-dec in-detail asked-Q \\
        Which student asked what the teacher ordered in detail?}
        \label{ex:twowh}
\end{exe}


\begin{center}
    \textbf{Example \ref{ex:twowh} presents an ambiguity to the parser.}
\end{center}
\vspace{0.2cm}
\fbox{%
    \parbox{\textwidth}{%
    \begin{center}
        ``\textcolor{red}{Dono} gakusei-ga kyoushi-ga \textcolor{red}{nani-o}...'' $\rightarrow$ is it 1 or 2 QPs coming?\\
        \vspace{0.3cm}
        chuumon-shita-to kuwashi-ku tazunemashita-\textcolor{red}{ka}'' \\(Just matrix QP)\\ 
        \vspace{0.2cm}
        chuumon-shita-\textcolor{red}{ka} kuwashi-ku tazunemashita-\textcolor{red}{ka}''\\ (Both matrix and embedded QP)
    \end{center}
    }%
}

\vspace{0.2cm}
\begin{center}
    \textbf{We can use this ambiguity to test for prediction.}
\end{center}

\blank
\newpage
\section{An Experiment}
Using the data from above we can make an experiment to test for predictive effects.
Assume the following conditions and an example item. Wh questions and QPs are highlighted red.

\begin{table}[H]
    \centering
    \begin{tabular}{cl|ll|}
    \cline{3-4}
    \multicolumn{1}{l}{}                               &      & \multicolumn{2}{c|}{\textbf{\# QP}} \\ \cline{3-4} 
    \textbf{}                                          &      & \multicolumn{1}{l|}{1}   & 2  \\ \hline
    \multicolumn{1}{|c|}{\multirow{2}{*}{\textbf{Matrix Wh}}} & Yes   & \multicolumn{1}{l|}{Cond. A }    & Cond. B    \\ \cline{2-4} 
    \multicolumn{1}{|c|}{}                             & No (Control) & \multicolumn{1}{l|}{Cond. C}    & Cond. D    \\ \hline
    \end{tabular}
    \caption{Conditions for Experiment}
    \label{table:exp2}
\end{table}

\begin{exe}    
    \ex{
        \begin{xlist}
        \ex{
        \gll \textcolor{red}{Dono} gakusei-ga [kyoushi-ga \textcolor{red}{nani-o} chuumon-shita-\circled{to}]
              kuwashi-ku tazunemashita-\textcolor{red}{ka}\\
        Which student-nom teacher-nom what-acc order-did-dec in-detail asked-Q \\
        Which student asked what the teacher ordered in detail?}

        \vspace{0.2cm}

        \ex{
        \gll \textcolor{red}{Dono} gakusei-ga [kyoushi-ga \textcolor{red}{nani-o} chuumon-shita-\textcolor{red}{\circled{ka}}]
              kuwashi-ku tazunemashita-\textcolor{red}{ka}\\
        Which student-nom teacher-nom what-acc order-did-Q in-detail asked-Q \\
        Which student asked what the teacher ordered in detail?}

        \vspace{0.2cm}

        \ex{
        \gll Sono gakusei-ga [kyoushi-ga \textcolor{red}{nani-o} chuumon-shita-to]
              kuwashi-ku tazunemashita-\textcolor{red}{ka}\\
        That student-nom teacher-nom what-acc order-did-dec in-detail asked-Q \\
        What did that student ask in detail that the teacher ate?}
        
        \vspace{0.2cm}

        \ex{
        \gll Sono gakusei-ga [kyoushi-ga \textcolor{red}{nani-o} chuumon-shita-\textcolor{red}{ka}]
              kuwashi-ku tazunemashita-\textcolor{red}{ka}\\
        That student-nom teacher-nom what-acc order-did-Q in-detail asked-Q \\
        What did that student ask in detail that the teacher ate?}
        \end{xlist}
    }
\end{exe}

\vspace{0.2cm}

\textbf{Prediction Hypothesis:}
\begin{itemize}
    \item When? $\rightarrow$ Upon reaching Wh (dono, nani-o)
    \item What? $\rightarrow$ A upcoming licensing QP
\end{itemize}

\vspace{0.2cm}
\newpage
\usetikzlibrary{topaths}
\tikzstyle{every picture}+=[remember picture,inner xsep=0,inner ysep=0.25ex]

\section{Expected Results}
Under predictive hypothesis, we hypothesize that the parser will do the following for conditions A/B.\\
\textcolor{red}{Red arrows indicate licensor $\rightarrow$ licensee relation.}
\begin{enumerate}
    \item Matrix ``Dono'' triggers prediction of upcoming QP licensor. First possible position is in Matrix C position.
    \begin{center}
        [_{CP} \tikz[baseline=(node1.base)]\node (node1) {Dono};
        ...\tikz[baseline=(node2.base)]\node (node2) {-ka};]
    \end{center}

    \item Embedded ``nani-o'' triggers prediction of upcoming licensor QP. First possible position is 
         embedded C. However, given matrix QP predicted in step 1, least effort is to assume nani-o is \textit{also} licensed by it. We assume parser will not predict new QPs if one can already be licensor.\\
          \begin{center}
            [_{CP} \tikz[baseline=(node5.base)]\node (node5) {Dono};
                    gakusei-ga[_{CP} kyoushi-ga \tikz[baseline=(node6.base)]\node (node6) {nani-o};...]
                    ...\tikz[baseline=(node7.base)]\node (node7) {-ka};]
        \end{center}

    \vspace{0.2cm}
    \item Parser continues onward through the sentence with this hypothesis.
    \item Parser reaches critical point, \hl{embedded C}.\\ Whatever comes up will confirm or deny its previous prediction (-to = not QP, -ka = QP)\\
    \begin{center}
        [_{CP} \tikz[baseline=(node3.base)]\node (node3) {Dono};
                gakusei-ga[_{CP} kyoushi-ga \tikz[baseline=(node4.base)]\node (node4) {nani-o}; chuumon-shita-\circled{\hl{??}}...]
                ...\tikz[baseline=(node12.base)]\node (node12) {-ka};]
    \end{center}

    \vspace{0.2cm}

    \item If \textbf{Condition A}, the embedded C has ``-to'', which is not a QP. Keep the same analysis as in 4.
    \item If \textbf{Condition B}, the embedded C has ``-ka'', which \textit{is} a QP. We should observe slowdown at its appearance due to reanalysis.\\
    \begin{center}
        [_{CP} \tikz[baseline=(node8.base)]\node (node8) {Dono};
               gakusei-ga[_{CP} kyoushi-ga \tikz[baseline=(node9.base)]\node (node9) {nani-o}; 
               chuumon-shita\tikz[baseline=(node10.base)]\node (node10) {\hl{-ka}};]
               ...\tikz[baseline=(node11.base)]\node (node11) {-ka};]
    \end{center}
\end{enumerate}

\begin{tikzpicture}[overlay]
    \draw[-latex,red] (node2.north) to[bend right] (node1.north);
    \draw[-latex,red] (node12.north) to[bend right=10] (node4.north);
    \draw[-latex,red] (node12.north) to[bend right=5] (node3.north);
    \draw[-latex,red] (node7.north) to[bend right=15] (node5.north);
    \draw[-latex,red] (node7.north) to[bend right=15] (node6.north);
    \draw[-latex,red] (node11.north) to[bend right=15] (node8.north);
    \draw[-latex,red] (node10.north) to[bend right=15] (node9.north);
\end{tikzpicture}

\begin{center}
\textbf{In condition B, a second (embedded) QP shows up. \\ Reanalysis of embedded WH licensing causes slowdown compared to A.}
\end{center}


\blank

\section{Concluding Thoughts}
Showing significant slowdown at the embedded QP in Condition B is in support of a structurally predictive parsing
model. \\

\textbf{More Questions:}
\begin{itemize}
    \item What other hypotheses could claim slowdown at embedded QP in Condition B?
    \item What about the \textit{how}?
        \begin{itemize}
            \item Hypothesis generation method?
            \item Stopping point?
        \end{itemize}
    \item Other Knowledge: Integrating frequency-based knowledge too?
    \item Limitations: Assuming what relations do we \textit{not} see predictive behavior?
    \item Acquisition: Learning to predict and predicting to learn?
\end{itemize}

\end{document}
